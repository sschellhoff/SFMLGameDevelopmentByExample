%%%%%%%%%%%%%%%%%%%%%%%%%%%%%%%%%%%%%%%%%
% Uppsala University Assignment Title Page 
% LaTeX Template
% Version 1.0 (27/12/12)
%
% This template has been downloaded from:
% http://www.LaTeXTemplates.com
%
% Original author:
% WikiBooks (http://en.wikibooks.org/wiki/LaTeX/Title_Creation)
% Modified by Elsa Slattegard to fit Uppsala university
% License:
% CC BY-NC-SA 3.0 (http://creativecommons.org/licenses/by-nc-sa/3.0/)

%\title{Title page with logo}
%----------------------------------------------------------------------------------------
%	PACKAGES AND OTHER DOCUMENT CONFIGURATIONS
%----------------------------------------------------------------------------------------

\documentclass[12pt]{article}

\usepackage{tikz,lipsum,lmodern}
\usepackage[most]{tcolorbox}

\usepackage[utf8]{inputenc}
\usepackage[greek,english]{babel}
\usepackage{alphabeta}
\usepackage{amsmath}
\usepackage{gfsartemisia}
\usepackage{graphicx}
\usepackage{subfig}
\usepackage{float}
\usepackage[colorinlistoftodos]{todonotes}
\usepackage{tabularx}
\usepackage[myheadings]{fullpage}
\usepackage{enumitem}
\PassOptionsToPackage{hyphens}{url}\usepackage{hyperref}
\usepackage{tikz}
\usepackage[nottoc]{tocbibind} %Adds "References" to the table of contents
\usepackage{xcolor} % to access the named colour LightGray
\definecolor{LightGray}{gray}{0.9}


\usepackage{tabto}
\usepackage{minted}


\hyphenpenalty=10000


\usepackage{geometry}
\geometry{
	a4paper,
	total={170mm,257mm},
	left=20mm,
	top=20mm,
}


\addto\captionsenglish{% Replace "english" with the language you use
	\renewcommand{\contentsname}%
	{Περιεχόμενα}%
}

\addto\captionsenglish{
	\renewcommand{\partname}{}
}
\renewcommand{\thepart}{}

\makeatletter
\renewcommand{\fnum@figure}{Εικόνα \thefigure}
\makeatother


\renewcommand{\H}{\textlozenge}

%=====fonts==========%
%\usepackage{libertine}


%====================%





%=============header + footer ======================%
%

\usepackage{fancyhdr}

\pagestyle{fancy}
\fancyhf{}
\rhead{Τεχνολογίες Ανάπτυξης Ηλεκτρονικών Παιχνιδιών \includegraphics[width=0.7cm]{UNIPI_(logo).png}}
\lhead{Πανεπιστήμιο Πειραιώς \\ Τμήμα Πληροφορικής}
\cfoot{Σελίδα \thepage}

% 

%\setlength\headheight{47pt}
%=====================================================%



%\setcounter{secnumdepth}{0} % sections are level 1

\begin{document}
	
	\begin{titlepage}
		
		\newcommand{\HRule}{\rule{\linewidth}{0.5mm}} % Defines a new command for the horizontal lines, change thickness here
		
		\center % Center everything on the page
		
		%----------------------------------------------------------------------------------------
		%	HEADING SECTIONS
		%----------------------------------------------------------------------------------------
		
		\textsc{\LARGE Πανεπιστήμιο Πειραιώς}\\[1.5cm] % Name of your university/college
		\includegraphics[scale=0.6]{UNIPI_(logo).png}\\[1cm] % Include a department/university logo - this will require the graphicx package
		\textsc{\Large Τμήμα Πληροφορικής}\\[0.5cm] % Major heading such as course name
		\textsc{\large Τεχνολογίες Ανάπτυξης Ηλεκτρονικών Παιχνιδιών}\\[0.5cm] % Minor heading such as course title
		
		%----------------------------------------------------------------------------------------
		%	TITLE SECTION
		%----------------------------------------------------------------------------------------
		
		\HRule \\[0.4cm]
		\textsc{\Large Ανάπτυξη 2d παιχνιδιού με SFML \\-\\ Creating a 2d game with SFML\\[0.4cm]} % Minor heading such as course title % Title of your document
		\HRule \\[1.5cm]
		
		%----------------------------------------------------------------------------------------
		%	AUTHOR SECTION
		%----------------------------------------------------------------------------------------
		%
		
		{  \bfseries ΌΝΟΜΑ: Αριστοτέλης Ματακιάς - Α.Μ: Π19100\\[0.4cm] Email Επικοινωνίας: mataktelis01@gmail.com\\[0.4cm] Επιβλέπων καθηγητής: Θεμιστοκλής Παναγιωτόπουλος\\[0.4cm] Ιανουάριος 2023}% Title of your document
		
		
		% If you don't want a supervisor, uncomment the two lines below and remove the section above
		%\Large \emph{Author:}\\
		%John \textsc{Smith}\\[3cm] % Your name
		
		%----------------------------------------------------------------------------------------
		%	DATE SECTION
		%----------------------------------------------------------------------------------------
		
		
		
		\vfill % Fill the rest of the page with whitespace
		
	\end{titlepage}
	
	
	
	%\selectlanguage{greek}
	\tableofcontents
	%\selectlanguage{english}
	
	\newpage
	
	 \section{Εισαγωγή}

	
	\newpage
	
	
	
	\section{Περιγραφή του προβλήματος}
	

	
	\section{Ανάλυση}
	
	\section{Σχεδίαση}
	
	\section{Υλοποίηση}
	
	\section{Εκτέλεση}
	
	
	\section{Κάλυψη κριτηρίων}
	
	\section{Συμπεράσματα και παρατηρήσεις}
	
	\section{Βιβλιογραφία}

	
	
	
\end{document}